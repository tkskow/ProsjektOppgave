\chapter{DNS Overview}
\label{chp:dns_intro}

\Gls{dns} is an important protocol for the internet. It is mostly used to translate a domain name to an IP address which the network use to route http traffic. This type of lookup receive an \texttt{'A'} record if the IP is an ipv4 address and \texttt{'AAAA'} if it's an ipv6 address. \texttt{'CNAME'} is also a much used response. it returns the correct domain name for the 'A' lookup, e.g. if you want to go to aftenposten.no, you could write ap.no the \Gls{dns} then respond with a CNAME response containing aftenposten.no which automatically trigger a new request for aftenposten.no which give an 'A' response containing the ipv4 address. There are over 30 different record types in the \Gls{dns}. Every one has their different purpose and therefore different maximum size on the payload. \Gls{dns} mostly use UDP on port 53, but could also use TCP on the same port. TCP is used when the payload is over 512 bytes or if there is a zone transfer. 

\Gls{dns} is build as a hierarchical system where each level sends you along until you have reached the correct server. The internet has 13 root servers, and a lookup in the system is backwards. The easiest way to explain this is with an example. If you request \texttt{some.test.example.com} the first request will be to the root server which will look up the IP-address of the server that controls the .com domain. Next the .com server looks up who controls the example.com domain, and the example.com server finds the \Gls{dns} server of test.example.com. At last the test.example.com \Gls{dns} server returns the IP-address of some.test.example.com. Since this process takes a long time, most responses has a \Gls{ttl} which is how long the router should use the given IP-address as a response to requests for that domain.


\cite{farnham2013detecting}
 