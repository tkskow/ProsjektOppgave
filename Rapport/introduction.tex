\chapter{Introduction}
\label{chp:intro}

\gls{dns} tunneling has been, and still is, an effective way to sneak around a pay wall to use internet at hotels and cafés for free. It is also a good way to send commands undetected in a \gls{cc} attack or exfiltrate data from a locked secure network. With the mobile network becoming an all-IP network \gls{dns} tunneling is becoming a threat to the mobile network as well. The fraudsters are able to use a tunnel to reduce the data traffic and use services without paying. For example, using the mobile phone as a wifi hotspot for a computer is a paid data service but by using a \gls{dns} tunnel on the computer is it possible to avoid the charge. There also exists tools to tunnel IP over \gls{dns} for mobile phones to avoid paying for the usage. This is not good for the carriers and unfair for the other users. The goal of this project is to study how \gls{dns} tunnel is done and how to detect it using machine learning. There are previously works on detection of \gls{dns} tunnels \cite{farnham2013detecting}, but there is still no solution that is really efficient. Further, mobile networks are also a new arena to this problem and might call for different solutions.



\section{Structure}
This report will start with a more complete overview of \gls{dns} in chapter \ref{chp:dns} and then explain how a \gls{dns} tunnel set up and used in chapter \ref{chp:dns_tunneling}. In chapter \ref{chp:dns_detection} there will be an overview of detection techniques previously tested. Next is an overview of machine learning in chapter \ref{chp:machine_learning}, in chapter \ref{chp:results} will the experiments be explained and the results be presented. Chapter \ref{chp:conclusion} contains the conclusion and thoughts about future work. 