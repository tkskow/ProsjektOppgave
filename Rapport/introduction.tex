\chapter{Introduction}
\label{chp:intro}

\gls{dns} tunneling has been, and still is, an effective way to sneak around a pay wall to use internet at hotels and cafés for free. It is also a good way to send commands undetected in a \gls{cc} attack or exfiltrate data from a locked secure network. With the mobile network becoming an all-IP network \gls{dns} tunneling is becoming a threat to the mobile network as well. The users is able to use a tunnel to reduce the data traffic and use services not paid for. If there is using your mobile as a hotspot is a service the carrier charges for using a \gls{dns} tunnel on your computer connected to the hotspot will not get you charged. There also exists tools for mobile phones to tunnel IP over \gls{dns} to get away from paying for data services completely. This is bad for the carriers, so in this project it is looked at how to detect a \gls{dns} tunnel using machine learning. Detecting \gls{dns} tunnels has been written about earlier \cite{farnham2013detecting}, but there is still no good way of doing it and the mobile network is a new scene for this problem. 



\section{Structure}
This report will start with a more complete overview of \gls{dns} in chapter \ref{chp:dns} and then explain how a \gls{dns} tunnel set up and used in chapter \ref{chp:dns_tunneling}. In chapter \ref{chp:dns_detection} will there be an overview of detection techniques previously tested. Next is an overview of machine learning in chapter \ref{chp:machine_learning}, in chapter \ref{chp:results} will the experiments be explained and the results be presented. Chapter \ref{chp:conclusion} contains the conclusion and thoughts about future work. 